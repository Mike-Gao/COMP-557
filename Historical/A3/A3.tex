\documentclass[12pt]{article}

 \usepackage{graphicx}
 \usepackage{amsmath}
 \usepackage{amssymb}
 \usepackage{algorithm}
 \usepackage{algorithmic}

\pagestyle{plain}

\begin{document}

\begin{center}
{\bf {\Large COMP 557 \hspace{1cm} Assignment 3} }
\end{center}

\begin{tabbing}
Instructor: \= \hspace{2cm} \= Prof. Michael Langer\\
Teaching Assistant: \> \> Marta Kersten  \\
Posted: \> \> Sat. Nov. 17, 2007   \\
Due:   \> \> Tues.  Dec. 4, 2007  (midnight) \\ \\
{\bf This assignment is worth 15 points.   Late policy:  2 points per day.} 
\end{tabbing}


\section*{3D Texturing for Volume Rendering}

Volume rendering is a useful technique for visualizing three
dimensional arrays of volumetric data obtained through sampling,
simulation or modeling techniques. In lecture 18, we looked at a
number of different techniques for visualizing volumetric data. In
this assignment you will create a simple volume renderer using
OpenGL's 3D texturing methods.


The 3D texture approach is a direct data visualization technique which
uses 2D or 3D textured data slices and combines the slices using
compositing (blending). The 3D texture approach produces equivalent
results to ray casting, where each image pixel is built up ray by ray.
However, it is much quicker because it takes advantage of the spatial
coherence of the volume. In 3D texturing, we "cast" one 2D layer or
slice at a time and blend the slices together.
% -- this results in a
%much faster algorithm than ray casting.

\section*{Requirements:}

You are given base code that sets up a 3D RGBA texture (your volume)
and a bounding box for your volume. Your task is to write the volume
rendering function, {\tt drawVolume()}, which renders the volume
within the bounding box.

The texture is $4 \times 4 \times 4$.  The colors of the four depth
layers are red, green, blue, and magenta (purple).  The opacity value
({\tt OPACITY}) is constant over the box, except that one of the
interior points is given an opacity of 1.  Texture values are
interpolated, and this single point with opacity 1 
results in a blurred ``mass'' inside the volume.

Here is the sequence of steps you must follow to render the volume:
\begin{enumerate}
\item  Find the closest and furthest vertex of the bounding box,
  relative to the camera position.  

\item For a given depth plane between the closest and furthest vertex,
  compute the polygon that is the intersection of this depth plane and
  the bounding volume.  By definition, this polygon is {\it
    perpendicular to the viewing direction}.  You can compute the
  vertices of the polygon by intersecting the plane with the edges of
  the bounding box.  You will use the resulting vertices for texture
  mapping.

  The RIGHT mouse buttom translates the eye
  position.  As the eye position changes, the view direction
  changes as well so that the eye always looks at the origin.  The
  result is that the orientation of each polygon slice will change as
  the eye moves so that the polygon remains perpendicular to the
  viewpoint.

  The LEFT mouse button uses {\tt glRotate()} to rotate the bounding
  box. The purpose of the rotation is to allow you to peek at the
  fixed polygon slice(s) from different viewpoints relative to the
  box.

%It does not update the view direction and therefore will not create 
%slices perpendicular to the view plane. 
Note that when you use the LEFT mouse button, the world coordinates of
the box change with the rotation, but the viewing position does not.
The rotation thus affects, for example, the distance to the closest
and furthest point on the box.  However, the rotation should not
affect the position and orientation of the polygon slices  which
  should remain fixed relative to the box.  (If this is unclear,
  please see me.)

Finally, problems with the compositing order can arise when the rotation is
  more than $\pm$ 90 degrees.  So, your solution only needs to work for
  rotations up to 90 degrees.

\item Sort the vertices of a polygon in either clockwise or
  counter-clockwise order so that the polygon is convex.  One way to
  do this is to pick a point in the interior of the polygon (e.g. the
  average position of the vertices) and then use the C function {\tt
    atan2} to compute the angle of each vertex relative to some chosen
  coordinate system with origin at this interior point.  Then, sort on this angle.

\item Choose texture coordinates to texture the polygon properly with
  respect to the 3D texture data.

\item Once you have the above working for one polygon, generalize it
   so that it works for a sequence of polygons.  You will need to
  enable and disable blending of the slices using {\tt
    GL\_BLEND}. \newline Use {\tt glBlendFunc(GL\_SRC\_ALPHA,
    GL\_ONE\_MINUS\_SRC\_ALPHA)} to composite the slices.

  To avoid complications with OpenGL's z buffer, you must
  render {\it back to front}.  Choose your slices so that they span
  the full range of depths -- from the furthest vertex to the closest.
  Note that this range of depths will change as the eye position
  changes.

%//(Hint: you
%can describe your plane as a point and a normal, and intersect each of
%the lines of the bounding box with the plane).

\end{enumerate} 

\noindent
  HINT: Use the constant {\tt NUMSLICES} to control how many polygon
  slices you use.  Begin by setting this constant to 1 and by setting
  the depth of your perpendicular plane to be midway between the
  closest and further vertex.



%Use {\tt
%    glBlendEquation( GL\_FUNC\_ADD )} and 
%{\tt glBlendFunc( GL\_ONE, GL\_ONE )} to  accumulate intensities.  


\section*{Hand-in Instructions:}

\begin{enumerate}
\item Hand in your files, including a README file that describes
any issues or limitations your code may have.
\item Use the handin command to submit the assignment, as in A1 and A2.
\item The programs will be tested on SOCS machines so if you are
  working from home then make sure you test it on the SOCS machines
  before handing it in.
\item You must properly comment your code.  
\item In addition to {\tt OpenGL: A Primer}, here are some good beginning references for the assignment: 
\begin{itemize}
\item  \begin{verbatim} http://www.opengl.org \end{verbatim}
\item  \begin{verbatim} http://nehe.gamedev.net/default.asp \end{verbatim}
\item  \begin{verbatim} http://fly.cc.fer.hr/~unreal/theredbook/ \end{verbatim}
\item  \begin{verbatim} http://www.lighthouse3d.com/opengl/glut/index.php?2 \end{verbatim}
\end{itemize}
\end{enumerate}


\section*{Starter Code:}
You do not have to set up the general graphics framework for the
assignment. In {\tt main.c} you will find the code that sets up the viewing window, adds mouse functionality etc. The code is set up to do perspective projection, note however that for orthographic you could simply replace gluPerspective with gluOrtho. 
You are also given an XYZ class (a point/vector class) which you can make use of.  The code already sets up a RGBA texture and the viewDir vector for you.  It also draws a cube (the bounding box of the volume) on screen. You should render your slices within the cube. 

You should take time to read the comments and understand how the graphics application is set up.

\end{document}